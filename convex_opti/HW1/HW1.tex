\documentclass[12pt,a4paper]{article}
\usepackage{datetime}
\newdate{date}{16}{10}{2023}
\date{\displaydate{date}}
\usepackage{pict2e,picture}
\makeatletter
\newcommand{\pnrelbar}{%
  \linethickness{\dimen2}%
  \sbox\z@{$\m@th\prec$}%
  \dimen@=1.1\ht\z@
  \begin{picture}(\dimen@,.4ex)
  \roundcap
  \put(0,.2ex){\line(1,0){\dimen@}}
  \put(\dimexpr 0.5\dimen@-.2ex\relax,0){\line(1,1){.4ex}}
  \end{picture}%
}
\newcommand{\precneq}{\mathrel{\vcenter{\hbox{\text{\prec@neq}}}}}
\newcommand{\prec@neq}{%
  \dimen2=\f@size\dimexpr.04pt\relax
  \oalign{%
    \noalign{\kern\dimexpr.2ex-.5\dimen2\relax}
    $\m@th\prec$\cr
    \noalign{\kern-.5\dimen2}
    \hidewidth\pnrelbar\hidewidth\cr
  }%
}
\makeatother
\usepackage{mathtools} % mathtools loads the amsmath package automatically
\usepackage[utf8]{inputenc}
\usepackage[french]{babel}
\usepackage[T1]{fontenc}
\usepackage{tikz}
\usepackage{amsmath}
\usepackage{amsfonts}
\usepackage{amssymb}
\usepackage{graphicx}
\usepackage{fourier}
\usepackage[left=2.5cm,right=2.5cm,top=2cm,bottom=2cm]{geometry}
\author{Théotime de Charrin}
\title{Convex Optimization - HW1}

\begin{document}
\maketitle
\date
\section{\textbf{Exercise 1} : Which of the following sets are convex?}
\paragraph{1)} The rectangle set defined as 
$ \left\lbrace x \in \mathbb{R}^n | ~\forall i \in  \llbracket 1, n \rrbracket , \alpha_i \leq x_i \leq \beta_i \right\rbrace$ \\
	For a given $i$, a $\left\lbrace x_i\right\rbrace$ is the intersection of two halfspaces $ \left\lbrace x \in \mathbb{R}^n 
| x_i \leq \beta_i \right\rbrace $ and  $\left\lbrace x \in \mathbb{R}^n |  x_i \geq \alpha_i \right\rbrace$ \\
As $i$ is finite, a rectangle is a finite intersection of halfspaces so it is a convex set.
\paragraph{2)} The hyperbolic set defined as $\left\lbrace H :  x \in R^2_+ | x_1x_2 \geq 1\right\rbrace$ .\\
We take $ x = \begin{pmatrix}
	x_1 \\ x_2  
\end{pmatrix},y = \begin{pmatrix}
	y_1\\y_2 
\end{pmatrix}\in R^2 $ such that $ x,y \in H $ and $ \theta \in [0;1] $
Hence, we have $ x_1 x_2 \geq 1 $ and the same for y.\\
Let's look at the convex combination of x and y : 
\begin{align*}
	\theta x + (1-\theta) y & = \begin{pmatrix}
		\theta x_1 + (1-\theta) y_1 \\
		\theta x_2 + (1-\theta) y_2
	\end{pmatrix} \\
				& = \begin{pmatrix}
					z_1 \\ z_2 
				\end{pmatrix}
\end{align*}
\[
	z_1 z_2 = \theta^2 \overbrace{x_1x_2}^{\geq 1}+ (1-\theta)^2 \overbrace{y_1y_2}^{\geq 1} + \theta (1-\theta) (x_1y_2+y_1x_2)
\]\newline
Let's have a look at $ x_1y_2 + y_1x_2 $ : we have $ x_1 x_2 \geq 1 \Leftrightarrow x_1 y_2 \geq \dfrac{y_2}{x_2} $. In the same manner, $ y_1 x_2 \geq \dfrac{x_2}{y_2} $.\\
We now have : \begin{align*}
	& x_1 y_2 + y_1 x_2 && \geq \dfrac{y_2}{x_2} + \dfrac{x_2}{y_2} \\
	& && \geq \dfrac{y_2}{x_2} + \dfrac{1}{\frac{y_2}{x_2}}\\
	\Leftrightarrow & x_1y_2 + y_1x_2 -2 &&\geq  \dfrac{y_2}{x_2} + \dfrac{1}{\frac{y_2}{x_2}}-2\\
			& && \geq \sqrt{\dfrac{y_2}{x_2}}^2+\dfrac{1}{\sqrt{\frac{y_2}{x_2}}}^2
		-2 \dfrac{\sqrt{\frac{y_2}{x_2}}}{\sqrt{\frac{y_2}{x_2}}}\\
			& && \geq \left( \sqrt{\dfrac{y_2}{x_2}}-\dfrac{1}{\sqrt{\dfrac{y_2}{x_2}}}\right)^2\geq 0 ~\forall \left\lbrace (x_1,x_2), (y_1,y_2)\right\rbrace
\end{align*}
We can conclude that $ x_1 y_2 + y_1 x_2 \geq 2 $\\
$ \theta (1-\theta)\geq 0 ~\forall \theta \in [0,1] $ so $ \theta (1-\theta) (x_1y_2+y_1x_2) \geq 2\theta (1-\theta) $ and : \begin{align*}
	z_1z_2 &\geq \theta ^2 + (1-\theta)^2 + 2\theta(1-\theta) \\
	       &\geq (\theta + 1 - \theta)^2 \geq 1
\end{align*}
We have shown that $ z_1z_2 \geq 1 $, i.e. $ \theta x + (1-\theta)y \in H $\\
A convex combination of two elements in H is also in H, so the hyperbolic set is convex as well.
\paragraph{3)} The set of points closer to a given point than a given set, \textit{i.e.} $$
	A = \left\lbrace x ~| ~{\| x - x_0 \|}_2 \leq {\|x-y\|}_2 ~\forall y \in S \right\rbrace \text{where} S \subseteq \mathbb{R}^n.
$$
Let  $ y \in S ~\text{and} ~x~\in A$. Then we must have :
\begin{align*}
&	{\| x- x_0 \|}_2 &&\leq {\| x-y\|}_2\\
			 &	(x-x_0)^T (x-x_0) &&\leq (x-y)^T(x-y) 			 \\
			 & (x^T - {x_0}^T )(x-x_0) &&\leq (x^T -y^T)(x-y)\\
			 & x^T x - x^T x_0 - {x_0}^Tx + {x_0}^T x_0 && \leq  x^Tx - x^Ty - y^Tx + y^Ty\\
			 & -2x^Tx_0 + {\|x_0\|}_2^2 &&\leq -2x^Ty + {\|y\|}_2^2\\
			 & 2x^T(y-x_0) &&\leq {\|x_0\|}_2^2 + {\|y\|}_2^2 \\
			 & (y-{x_0})^Tx &&\leq \dfrac{{\|x_0\|}_2^2 + {\|y\|}_2^2}{2}
\end{align*}
This is the equation of an halfspace ( $ a^T x \leq b $), which is a convex set. 
S is the intersection over $ y \in S $ of these halfspaces, hence it's also a convex set.
\paragraph{4)} The set of points closer to one set than another, \textit{i.e.} $$ A = \left\lbrace x ~| ~\text{\textbf{dist}}(x,S) \leq \text{\textbf{dist}}(x,T)\right\rbrace , \\ 
~\text{where} ~S,T \subseteq \mathbb{R}^n, ~\text{and} ~\text{\textbf{dist}}(x,S)=~\text{inf}\left\lbrace~{\|x-z\|}_2 ~| ~z \in S\right\rbrace$$
A visual example should better explain why this is not always a convex set :
\begin{center}
\begin{tikzpicture}
	 \draw[step=1cm,gray,very thin] (-2,-2) grid (7,7);
 \filldraw[fill=blue!40!white, draw=black] (-2,1) rectangle (2,5);
 \filldraw[fill=red!40!white, draw=black] (5,2) rectangle (7,4);
 \node[draw] at (0,3) {set S};
 \node[draw] at (6,3) {set T};
 \draw[fill=black] (4,7) circle (0.05) node[below left] {$ x_1 $};
 \draw[fill=black] (4,-1) circle (0.05) node[above right] {$ x_2 $};
 \draw[fill=black] (4,3) circle (0.05) node[above right] {$ x_3 $};
	 %\draw[blue,thick] (5,5) rectangle (6,6);
\end{tikzpicture}
\end{center}
Here, we see that $ x_1 ~\text{and}~ x_2 $ are closer to the set S rather than the set T. Therefore, they would be part of the set A. 
However, $ x_3 $, a convex combination of $ x_1,~x_2 $, \textit{i.e.} on the same segment, is closer to the set T than the set S.\\
Therefore, A is not a convex set.
\paragraph{5)} The set $$A= \left\lbrace x~|~x+S_2 ~\subseteq S_1\right\rbrace $$ where $ S_1,~S_2~\in \mathbb{R}^n $,$ ~S_1 $ convex.\\
\begin{equation*}
	x+S_2 \subseteq S_1 \Leftrightarrow \forall y~\in S_2,~x+y\in S_1
\end{equation*}
\\
Let's look at the convex combination of $ u,~v \in A,~\theta \in [0,1] $, and see if it's still in A :\\
\begin{align*}
	\theta u + (1-\theta)v  + y = \theta(u+y) + (1-\theta)(v+y) \in S_1
\end{align*}
By definition, $ u+y,~v+y \in S_1 $ because $ u,~v \in A~\text{and}~y \in S_2 $. \\
As $ S_1 $ is convex, any convex combination of $ z \in S_1 $ is also in $ S_1 $. 
We conclude that A is a convex set.
\section{\textbf{Exercise 2 : }For each of the following functions determine whether it is convex or concave or not.}
\paragraph{1)} $ f(x_1,x_2)=x_1x_2 $ on $ \mathbb{R}^2_{++} $
\begin{itemize}
	\item $ \text{dom} ~f=\mathbb{R}^2_{++} $ which is convex.
	\item f is twice differentiable, let's look at the second-order conditions :\\

 \end{itemize}
 \begin{center}
$$	\begin{cases}
			\dfrac{\partial f}{\partial x_1} &=  x_2 \\
			\dfrac{\partial f}{\partial x_2} &=  x_1 
		\end{cases}
		\quad 
		\Rightarrow
		\quad
		\left\lbrace
		\begin{matrix}
			\dfrac{\partial^2 f}{\partial^2 x_1} =  0 \quad 
			&\dfrac{\partial^2 f}{\partial^2 x_1x_2} =  1 \\
			\dfrac{\partial^2 f}{\partial^2 x_2} =  0 \quad  
							     &\dfrac{\partial^2 f}{\partial^2 x_2x_1} =  1 
		\end{matrix} \right.	$$
	\end{center}
	Hence, we have the following hessian : \def\arraystretch{2} $ \Delta^2 f(x_1,x_2)=	\begin{pmatrix}
		0 & 1 \\
		1 & 0	 
        \end{pmatrix} $\\
With -1 and 1 as eigen values. $ \Delta^2 f \notin S_n^{+} $, so f is not convex. In the same manner, $\Delta^2 f \npreceq 0$
\\ We conclude that F is neither convex nor concave\\
\paragraph{2)} $ f(x_1,x_2)=\frac{1}{x_1x_2} ~\text{on}~ \mathbb{R}_{++}^2$

\begin{itemize}
	\item $ \text{dom} ~f=\mathbb{R}^2_{++} $ which is convex.
	\item f is twice differentiable, let's look at the second-order conditions :\\
 \end{itemize}
 \begin{center} 
$$	\begin{cases}
	\dfrac{\partial f}{\partial x_1} &= -\dfrac{1}{x_1^2x_2} \\
	\dfrac{\partial f}{\partial x_2} &= -\dfrac{1}{x_2^2x_1}
		\end{cases}
		\quad 
		\Rightarrow
		\quad
		\left\lbrace
		\begin{matrix}
			\dfrac{\partial^2 f}{\partial^2 x_1} =  \dfrac{2}{x_1^3x_2} \quad 
			&\dfrac{\partial^2 f}{\partial^2 x_1x_2} =  \dfrac{1}{x_1^2x_2^2} \\
			\dfrac{\partial^2 f}{\partial^2 x_2} =  \dfrac{2}{x_2^3x_1} \quad 
			&\dfrac{\partial^2 f}{\partial^2 x_2x_1} =  \dfrac{1}{x_1^2x_2^2}
		\end{matrix} \right.	$$
	\end{center}
	Hence, we have the following hessian : \def\arraystretch{2} $ \Delta^2 f(x_1,x_2)=	\begin{pmatrix}
		\dfrac{2}{x_1^3x_2} & \dfrac{1}{x_1^2x_2^2} \\
		\dfrac{1}{x_1^2x_2^2} & \dfrac{2}{x_2^3x_1}
        \end{pmatrix} $\\
        
        $
	\begin{cases}
		\text{\textbf{det}}(\Delta^2f)= \dfrac{4}{x_1^4x_2^4}-\dfrac{1}{x_1^4x_2^4} > 0\\
		\text{Tr}(\Delta^2f)=\dfrac{2x_2^2+2x_1^2}{x_1^3x_2^3}>0
	\end{cases}
$
		$\quad \Rightarrow \quad \lambda_1\lambda_2>0,~\lambda_1+\lambda_2>0~: $All diagonal values are positive,  the determinant is positive so all eigenvalues are positive, hence $~\Delta^2f$ is a positive definite matrix, $~\Delta^2f \succ 0$.
		We can say that f is a convex function on $ \mathbb{R}^2_{++}. $	
		\paragraph{3)} $ f=\frac{x_1}{x_2} ~\text{on} ~\mathbb{R}^2_{++} $

\begin{itemize}
	\item $ \text{dom} ~f=\mathbb{R}^2_{++} $ which is convex.
	\item f is twice differentiable, let's look at the second-order conditions :\\
 \end{itemize}
 \begin{center} 
$$	\begin{cases}
	\dfrac{\partial f}{\partial x_1} &= \dfrac{1}{x_2} \\
	\dfrac{\partial f}{\partial x_2} &= -\dfrac{x_1}{x_2^2}
		\end{cases}
		\quad 
		\Rightarrow
		\quad
		\left\lbrace
		\begin{matrix}
			\dfrac{\partial^2 f}{\partial^2 x_1} = 0 
			&\dfrac{\partial^2 f}{\partial^2 x_1x_2} =  -\dfrac{1}{x_2^2} \\
			\dfrac{\partial^2 f}{\partial^2 x_2} =  -\dfrac{2x_1}{x_2^3} \quad 
			&\dfrac{\partial^2 f}{\partial^2 x_2x_1} =  \dfrac{1}{x_2^2}
		\end{matrix} \right.	$$
	\end{center}
	Hence, we have the following hessian : \def\arraystretch{2}  $ \Delta^2 f(x_1,x_2)=	\begin{pmatrix}
		0	
			&-\dfrac{1}{x_2^2} \\ \bigskip
			 -\dfrac{1}{x_2^2}
			& \dfrac{2x_1}{x_2^3} 
        \end{pmatrix} $\\
	$\begin{cases} \text{\textbf{det}}~(\Delta^2 f )= \overbrace{-\dfrac{2x_1}{x_2^3}}^{<0, ~\text{dom} ~f =~ 
	\mathbb{R}^2_{++}}-{\dfrac{1}{x_2^2}}^2 < 0 \\
	\text{Tr}~(\Delta^2f)~=\dfrac{2x_1}{x_2^3}>0 
\end{cases} $
		$\quad \Rightarrow \quad \lambda_1\lambda_2<0,~\lambda_1+\lambda_2>0~: $All diagonal values are positive,  the determinant is negative so one eigenvalue is positive an the other negative, hence $~\Delta^2f \npreceq 0$ , $~\Delta^2f \nsucceq 0$.
\\ We can conclude that f is neither convex nor concave.
\paragraph{4)} $ f(x_1,x_2)=x_1^{\alpha} x_2^{1-\alpha} $, where $ 0 \leq \alpha \leq 1,~\text{on}~ \mathbb{R}^2_{++} $
\begin{itemize}
	\item $ \text{dom} ~f=\mathbb{R}^2_{++} $ which is convex.
	\item f is twice differentiable, let's look at the second-order conditions :\\
 \end{itemize}
 \begin{center} 
$$	\begin{cases}
	\dfrac{\partial f}{\partial x_1} &=\alpha x_1^{\alpha -1}x_2^{1-\alpha} \\
	\dfrac{\partial f}{\partial x_2} &=(1-\alpha) x_1^{\alpha}x_2^{-\alpha} 
		\end{cases}
		\quad 
		\Rightarrow
		\quad
		\left\lbrace
		\begin{matrix}
			\dfrac{\partial^2 f}{\partial^2 x_1} = \alpha (\alpha -1)x_1^{\alpha-2}x_2^{1-\alpha} 
			&\dfrac{\partial^2 f}{\partial^2 x_1x_2} =\alpha (1-\alpha)x_1^{\alpha -1}x_2^{-\alpha}\\ 
			\dfrac{\partial^2 f}{\partial^2 x_2} = \alpha(\alpha-1)x_1^{\alpha}x_2^{-\alpha - 1} 
			&\dfrac{\partial^2 f}{\partial^2 x_2x_1} = \alpha (1-\alpha)x_1^{\alpha -1}x_2^{-\alpha}
		\end{matrix} \right.	$$
	\end{center}
	Hence, we have the following hessian : \def\arraystretch{2}  $ \Delta^2 f(x_1,x_2)=	\begin{pmatrix}
			 \alpha (\alpha -1)x_1^{\alpha-2}x_2^{1-\alpha} 
			&\alpha (1-\alpha)x_1^{\alpha -1}x_2^{-\alpha}\\ 
			\alpha (1-\alpha)x_1^{\alpha -1}x_2^{-\alpha}
			&\alpha(\alpha-1)x_1^{\alpha}x_2^{-\alpha - 1} \\
        \end{pmatrix} $\\
	$$\begin{cases} \text{\textbf{det}}~(\Delta^2 f )&= 
			 \alpha (\alpha -1)x_1^{\alpha-2}x_2^{1-\alpha} \times 
			\alpha(\alpha-1)x_1^{\alpha}x_2^{-\alpha - 1} -
			\alpha (1-\alpha)x_1^{\alpha -1}x_2^{-\alpha}\times 
			\alpha (1-\alpha)x_1^{\alpha -1}x_2^{-\alpha}\\
			&=\alpha^2(\alpha-1)^2 \left[ x_1^{2\alpha -2}x_2^{-2\alpha}+x_1^{2\alpha -2}x_2^{-2\alpha}\right]\\
			&=\alpha^2(\alpha-1)^2\left(\dfrac{x_1^{\alpha-1}}{x_2^{\alpha}}\right)^2\geq 0\\
			\text{Tr}~(\Delta^2f)~&=\underbrace{\alpha(\alpha-1)}_{\leq 0}\left(x_1^{\alpha-2}x_2^{1-\alpha}+x_1^{\alpha}x_2^{-\alpha-1}\right)\leq 0
\end{cases} $$\\
		$\quad \Rightarrow \quad \lambda_1\lambda_2\geq0,~\lambda_1+\lambda_2\leq0~: $ All diagonal values are negative,
		the determinant is positive so all eigenvalues are negatives, hence $~\Delta^2f \preceq 0$ .\\
We can conclude that f is concave. 
\section{Exercise 3 : Show that following functions are convex}
\paragraph{1)} $ f(X)=\text{\textbf{Tr}}(X^{-1})~\text{on \textbf{dom}}~f=S^n_{++}. $
\begin{itemize}
	\item $ \text{dom} ~f=S^2_{++} $ which is convex.
	\item f is twice differentiable, the second-order conditions might apply:\\
 \end{itemize}
 Let's look at the function $ g(t)=f(X+tV), ~\text{dom}X=\{t~|~X+tV \in \text{dom} f\}$.
 \\Let's pose $ X \in S^n_{++} \text{and} V \in S_n $. If for all X and V in this domain g is convex, 
 then f(X) must be convex as well.\\
\begin{align*}
	g(t) &= \text{Tr}((X+tV)^{-1})\\
	\text{Let's look inside the Trace :}~ &= \text{Tr}(X(I+tX^{-1}V))^{-1})\\
	\text{Let's make a Taylor expansion in t} : ~&= \text{Tr}(X^{-1}-tX^{-1}VX^{-1}+t^2 +t^2X^{-1}VX^{-1}VX^{-1} + \dots)
\end{align*}
To show that Tr($ X^{-1} $) is convex, we can look at the second derivative, when t=0:
\begin{align*}
	\left. \dfrac{\partial^2g}{\partial t^2} \right|_{t=0} = 2\text{Tr}(X^{-1}VX^{-1}VX^{-1})
\end{align*}
We can write this as $ PX^{-1}P^T $, with $ P=X^{-1}V $, and $ X^{-1} \in S^n_{++} $.
Hence $ PX{-1}P^T $ is positive definite as well, so its trace must be positive.\\
%	It comes that $\dfrac{\partial^2g}{\partial t^2} \geq 0$, \textit{i.e.} g is convex. f(X) must be convex as well.
%	\paragraph{2)} $ f(X,y)=y^T X^{-1}y,~\text{dom}f=S^n_{++}\times \mathbb{R}^n $\\
%	According to lecture slides (24/67, Functions and problems). We have \\ $ \dfrac{1}{2} y^TQ^{-1}y = ~\text{sup}(y^Tx - \dfrac{1}{2}x^TQx  $,
%	with x another column vector in $ \mathbb{R}^n $.
%	\\ 	It comes for our problem that : $ y^TX^{-1}y=~\text{sup}~ (2y^Tx-x^TXx)$
%	\\ Let's look at the function g(X,y) which is inside the trace : \begin{align*}
%		f(X,y)&=sup(g(X,y))\\
%		g(X,y)&= 2y^Tx - x^TXx) 
%	\end{align*}
	g is a linear function of y and X, hence a convex function in (X,y)\\
The supremum of a set of convex functions (every g(X,y), with any x in $ \mathbb{R}^n $ )is a convex function.\\
We conclude that f(X,y) is a convex function.
\paragraph{3)} $ f(X)= \sum_{i=1}^{n}\sigma_i(X)$ on \textbf{dom} f $ = S^n $, where $ \sigma_1(X),~\dots,\sigma_n(X) $ are the singular values of a matrix $ X \in \mathbb{R}^{n \times n } $ 
\begin{itemize}
	\item $ \text{dom} ~f=S^n $ which is convex.
 \end{itemize}
f is also known as the nuclear norm. A norm is always convex, so we just have to show that f is a norm.\\
We want to verify the definition of norm : \\
\begin{itemize}
	\item Homogeneity : $\| \alpha A\| = |\alpha| \times \| A\ $
for $ \alpha \in \mathbb{R} $\\
\item Triangular Inequality: $ \| A+B\|\leq \|A\|+\|B\|$\\
\item Positive definiteness: if $\|A\|=0$ ,
 then A=0\\
\item Positivity : $ \|A\|\geq 0 $\\
\end{itemize}
Regarding the positivity :\\
\[
	\|A\|=\sum_1^n \underbrace{\sigma_i(A)}_{\geq 0} \geq 0
\]
Regarding the positive definiteness: 
$$
\begin{cases}
	\|A\|=0 \Rightarrow \sum_{i=1}^{n} \underbrace{\sigma_i}_{\geq 0} =0 \Rightarrow \sigma_i=0 \forall i \Rightarrow A=0.\\
A=0 \Rightarrow \sigma_i=0 \forall i \Rightarrow \|A\|=0	
\end{cases}
$$
Regarding the homogeneity, for t a scalar :
\begin{align*}
	\|tA\| = \sum_{i=1}^{n}\sigma_i (tA) &\underbrace{=}_{\text{As A is in}~S^n} 
	\sqrt{\|t^2\|\lambda_i}^2\\
	       &= |t| \sum_1^n \sqrt{{\lambda_i}^2}\\
	       &= |t| \sum_1^n \sigma_i (A)\\
	       &= |t| \|A\|
\end{align*}
Regarding the triangular inequality :\\
Using the supremum, we will try to prove that \[ \sup_{\sigma_1(Q)\leq 1} \langle Q,A \rangle = 
\sup_{\sigma_1(Q)\leq 1} \text{Tr}(Q^TA)=\sum_1^n \sigma_i(A)=\|A\| 
\] Where $ \sigma_i(\cdot) $ is itself a norm, and $ \sigma_1 $ is the maximum singular value of Q. \\
Let $ A = U\Sigma V^T = \sum \sigma_i u_i v_i^T $ be the singular value decomposition of A, and pose $ Q=UV^T = UIV^T$. 
Hence, $ \sigma_1(Q)=1 $.\\
\begin{align*}
	\langle Q,A \rangle &= \langle UV^T, U\Sigma V^T \rangle \\
			    &= \text{Tr}(VU^TU\Sigma V^T)\\
			    &= \text{Tr}(V^TVU^TU\Sigma)
			    &= \text{Tr}(\Sigma)
\end{align*}
Hence, it comes that \[
	\sup_{\sigma_1(Q)\leq 1} \langle Q,A \rangle \geq \sum_1^n \sigma_i(A)
\]
Let's prove the other direction and we will have an equality. \begin{align*}
	\sup_{\sigma_1(Q)\leq 1} \langle Q,A \rangle &= \sup_{\sigma_1(Q)\leq1} \text{Tr}(Q^TA) \\
						    &= \sup_{\sigma_1(Q)\leq 1}\text{Tr}(Q^TU\Sigma V^T)\\
						    &= \sup_{\sigma_1(Q)\leq 1}\text{Tr}(V^TQ^TU\Sigma)\\
						    &= \sup_{\sigma_1(Q)\leq 1}\langle U^TQV,\Sigma \rangle \\
						    &= \sup_{\sigma_1\leq 1}\sum_1^n \sigma_i(U^TQV)\\
						    &= \sup_{\sigma_1 \leq 1} \sum_1^n \sigma_i u_i^T Q v_i \\
						    & \leq \sup_{\sigma_1 \leq 1} \sum_1^n \sigma_i \sigma_{max}(Q)\\
						    &= \sum_i^n \sigma_i (A) = \|A\|	 
\end{align*}
We have proven the equality $ \sup_{\sigma_1(Q)\leq 1} = \|A\| $. We can now show the triangular inequality : \begin{align*}
	\|A+B\| &= \sup_{V, \sigma_1(V)\leq 1} \langle V, A+B \rangle \\
	&\leq \sup_{V,~\sigma_1(V)\leq 1}\langle V, A \rangle + \langle V, B \rangle = \|A\|+\|B\|.
\end{align*}
 
Hence, we've shown that the nuclear norm is a norm, so f is convex.
\end{document}
