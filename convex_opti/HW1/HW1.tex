\documentclass[12pt,a4paper]{article}
\usepackage[utf8]{inputenc}
\usepackage[french]{babel}
\usepackage[T1]{fontenc}
\usepackage{tikz}
\usepackage{amsmath}
\usepackage{amsfonts}
\usepackage{amssymb}
\usepackage{graphicx}
\usepackage{fourier}
\usepackage[left=2.5cm,right=2.5cm,top=2cm,bottom=2cm]{geometry}
\author{Théotime de Charrin}
\title{Convex Optimization - HW1}

\begin{document}
\maketitle
\section{\textbf{Exercise 1} : Which of the following sets are convex?}
\paragraph{1)} The rectangle set defined as 
$ \left\lbrace x \in \mathbb{R}^n | ~\forall i \in  \llbracket 1, n \rrbracket , \alpha_i \leq x_i \leq \beta_i \right\rbrace$ \\
	For a given $i$, a $\left\lbrace x_i\right\rbrace$ is the intersection of two halfspaces $ \left\lbrace x \in \mathbb{R}^n 
| x_i \leq \beta_i \right\rbrace $ and  $\left\lbrace x \in \mathbb{R}^n |  x_i \geq \alpha_i \right\rbrace$ \\
As $i$ is finite, a rectangle is a finite intersection of halfspaces so it is a convex set.
\paragraph{2)} The hyperbolic set defined as $\left\lbrace H :  x \in R^2_+ | x_1x_2 \geq 1\right\rbrace$ .\\
We take $ x = \begin{pmatrix}
	x_1 \\ x_2  
\end{pmatrix},y = \begin{pmatrix}
	y_1\\y_2 
\end{pmatrix}\in R^2 $ such that $ x,y \in H $ and $ \theta \in [0;1] $
Hence, we have $ x_1 x_2 \geq 1 $ and the same for y.\\
Let's look at the convex combination of x and y : 
\begin{align*}
	\theta x + (1-\theta) y & = \begin{pmatrix}
		\theta x_1 + (1-\theta) y_1 \\
		\theta x_2 + (1-\theta) y_2
	\end{pmatrix} \\
				& = \begin{pmatrix}
					z_1 \\ z_2 
				\end{pmatrix}
\end{align*}
\[
	z_1 z_2 = \theta^2 \overbrace{x_1x_2}^{\geq 1}+ (1-\theta)^2 \overbrace{y_1y_2}^{\geq 1} + \theta (1-\theta) (x_1y_2+y_1x_2)
\]\newline
Let's have a look at $ x_1y_2 + y_1x_2 $ : we have $ x_1 x_2 \geq 1 \Leftrightarrow x_1 y_2 \geq \dfrac{y_2}{x_2} $. In the same manner, $ y_1 x_2 \geq \dfrac{x_2}{y_2} $.\\
We now have : \begin{align*}
	& x_1_y_2 + y_1x_2 && \geq \dfrac{y_2}{x_2} + \dfrac{x_2}{y_2} \\
	& && \geq \dfrac{y_2}{x_2} + \dfrac{1}{\frac{y_2}{x_2}}\\
	\Leftrightarrow & x_1y_2 + y_1x_2 -2 &&\geq  \dfrac{y_2}{x_2} + \dfrac{1}{\frac{y_2}{x_2}}-2\\
			& && \geq \sqrt{\dfrac{y_2}{x_2}}^2+\dfrac{1}{\sqrt{\frac{y_2}{x_2}}}^2
		-2 \dfrac{\sqrt{\frac{y_2}{x_2}}}{\sqrt{\frac{y_2}{x_2}}}\\
			& && \geq \left( \sqrt{\dfrac{y_2}{x_2}}-\dfrac{1}{\sqrt{\dfrac{y_2}{x_2}}}\right)^2\geq 0 ~\forall \left\lbrace (x_1,x_2), (y_1,y_2)\right\rbrace
\end{align*}
We can conclude that $ x_1y_2 + y_1x_2 \geq 2 $\\
$ \theta (1-\theta)\geq 0 ~\forall \theta \in [0,1] $ so $ \theta (1-\theta) (x_1y_2+y_1x_2) \geq 2\theta (1-\theta) $ and : \begin{align*}
	z_1z_2 &\geq \theta ^2 + (1-\theta)^2 + 2\theta(1-\theta) \\
	       &\geq (\theta + 1 - \theta)^2 \geq 1
\end{align*}
We have shown that $ z_1z_2 \geq 1 $, i.e. $ \theta x + (1-\theta)y \in H $\\
A convex combination of two elements in H is also in H, so the hyperbolic set is convex as well.
\paragraph{3)} The set of points closer to a given point than a given set, \textit{i.e.} $$
	A = \left\lbrace x ~| ~{\| x - x_0 \|}_2 \leq {\|x-y\|}_2 ~\forall y \in S \right\rbrace \\ \text{where} S \subseteq \mathbb{R}^n.

$$
Let  $ y \in S ~\text{and} ~x~\in A$. Then we must have :
\begin{align*}
&	{\| x- x_0 \|}_2 &&\leq {\| x-y\|}_2\\
			 &	(x-x_0)^T (x-x_0) &&\leq (x-y)^T(x-y) 			 \\
			 & (x^T - {x_0}^T )(x-x_0) &&\leq (x^T -y^T)(x-y)\\
			 & x^T x - x^T x_0 - {x_0}^Tx + {x_0}^T x_0 && \leq  x^Tx - x^Ty - y^Tx + y^Ty\\
			 & -2x^Tx_0 + {\|x_0\|}_2^2 &&\leq -2x^Ty + {\|y\|}_2^2\\
			 & 2x^T(y-x_0) &&\leq {\|x_0\|}_2^2 + {\|y\|}_2^2 \\
			 & (y-{x_0})^Tx &&\leq \dfrac{{\|x_0\|}_2^2 + {\|y\|}_2^2}{2}
\end{align*}
This is the equation of an halfspace ( $ a^T x \leq b $), which is a convex set. 
S is the intersection over $ y \in S $ of these halfspaces, hence it's also a convex set.
\paragraph{4)} The set of points closer to one set than another, \textit{i.e.} $$ A = \left\lbrace x ~| ~\text{\textbf{dist}}(x,S) \leq \text{\textbf{dist}}(x,T)\right\rbrace , \\ 
~\text{where} ~S,T \subseteq \mathbb{R}^n, ~\text{and} ~\text{\textbf{dist}}(x,S)=~\text{inf}\left\lbrace~{\|x-z\|}_2 ~| ~z \in S\right\rbrace$$
A visual example should better explain why this is not always a convex set :
\begin{center}
\begin{tikzpicture}
	 \draw[step=1cm,gray,very thin] (-2,-2) grid (7,7);
 \filldraw[fill=blue!40!white, draw=black] (-2,1) rectangle (2,5);
 \filldraw[fill=red!40!white, draw=black] (5,2) rectangle (7,4);
 \node[draw] at (0,3) {set S};
 \node[draw] at (6,3) {set T};
 \draw[fill=black] (4,7) circle (0.05) node[below left] {$ x_1 $};
 \draw[fill=black] (4,-1) circle (0.05) node[above right] {$ x_2 $};
 \draw[fill=black] (4,3) circle (0.05) node[above right] {$ x_3 $};
	 %\draw[blue,thick] (5,5) rectangle (6,6);
\end{tikzpicture}
\end{center}
Here, we see that $ x_1 ~\text{and}~ x_2 $ are closer to the set S rather than the set T. Therefore, they would be part of the set A. 
However, $ x_3 $, a convex combination of $ x_1,~x_2 $, \textit{i.e.} on the same segment, is closer to the set T than the set S.\\
Therefore, A is not a convex set.
\paragraph{5)} The set $$A= \left\lbrace x~|~x+S_2 ~\subseteq S_1\right\rbrace $$ where $ S_1,~S_2~\in \mathbb{R}^n $,$ ~S_1 $ convex.\\
\begin{equation*}
	x+S_2 \subseteq S_1 \Leftrightarrow \forall y~\in S_2,~x+y\in S_1
\end{equation*}
\\
Let's look at the convex combination of $ u,~v \in A,~\theta \in [0,1] $, and see if it's still in A :\\
\begin{align*}
	\theta u + (1-\theta)v  + y = \theta(u+y) + (1-\theta)(v+y) \in S_1
\end{align*}
By definition, $ u+y,~v+y \in S_1 $ because $ u,~v \in A~\text{and}~y \in S_2 $. \\
As $ S_1 $ is convex, any convex combination of $ z \in S_1 $ is also in $ S_1 $. 
We conclude that A is a convex set.
\section{\textbf{Exercise 2 : }For each of the following functions determine whether it is convex or concave or not.}
\paragraph{1)} $ f(x_1,x_2)=x_1x_2 $ on $ \mathbb{R}^2_{++} $
\\ 
\end{document}
